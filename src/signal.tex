\section{Зачем искать события ++ }

\begin{frame}{Поиск событий: временные ряды}

\begin{columns}[t]
  \begin{column}{0.5\textwidth}
        \bluebox{Изначально имеем }{
            \scriptsize
            \begin{itemize}
                \item последовательность изображений;
                \item и последовательность аудио-отчетов.
            \end{itemize}
        }
    \end{column}
    \begin{column}{0.5\textwidth}
        \orangebox{{Хотим получить}}{
            \scriptsize
            \begin{itemize}
                \item временной ряд;
                \item описание ситуации в видео.
            \end{itemize}
        }
    \end{column}
\end{columns}
\begin{tikzcd}[column sep=1.5pc,ampersand replacement=\&]
    \raisebox{-.4\height}{
        \begin{tikzpicture}[scale = 1.6,
            x={(-0.5cm,-0.04cm)},
            y={(0.3cm,-0.09cm)},
            z={(0cm,0.5cm)},]
        \tikzset{normalframe/.style={
            fill=black!3,opacity=0.5,draw=black
        }}
        \draw[->,>=stealth,thick,black,dashed,line join=round]
            (1.5,-1,1) -- (1.5,0,1);
        \begin{scope}[canvas is zx plane at y=0]
            \path[normalframe] (0,0) rectangle (2,3);
            \draw[black] (2,0) node [above] {{\footnotesize 
                последовательность изображений
            }};
        \end{scope}
        \draw[->,>=stealth,thick,black,dashed,line join=round]
            (1.5,0,1) -- (1.5,1,1);
        \begin{scope}[canvas is zx plane at y=1]
            \path[normalframe] (0,0) rectangle (2,3);
        \end{scope}
        \draw[->,>=stealth,thick,black,dashed,line join=round]
            (1.5,1,1) -- (1.5,2,1);
        \begin{scope}[canvas is zx plane at y=2]
            \path[normalframe] (0,0) rectangle (2,3);
        \end{scope}
        \draw[->,>=stealth,thick,black,dashed,line join=round]
            (1.5,2,1) -- (1.5,3,1);
        \begin{scope}[canvas is zx plane at y=3]
            \path[normalframe] (0,0) rectangle (2,3);
        \end{scope}
        \draw[->,>=stealth,thick,black,dashed,line join=round]
            (1.5,3,1) -- (1.5,4,1);
        \begin{scope}[canvas is zx plane at y=4]
            \path[normalframe] (0,0) rectangle (2,3);
        \end{scope}
        \draw[->,>=stealth,thick,black,dashed,line join=round]
            (1.5,4,1) -- (1.5,5,1) node [sloped, below left] {$t$} ;
        \draw [teal,thick,decorate, decoration={
            coil,
            aspect=0,
            segment length=5,
            amplitude=2.0,
        }]  (3.1,-0.5,1) -- (3.1,0,1);
        \draw [teal,thick,decorate, decoration={
            coil,
            aspect=0,
            segment length=4,
            amplitude=5.0,
        }]  (3.1,0,1) -- (3.1,1,1);
       \draw [teal,thick,decorate, decoration={
           coil,
           aspect=0,
           segment length=3,
           amplitude=1.0,
        }]  (3.1,1,1) -- (3.1,2,1);
       \draw [teal,thick,decorate, decoration={
           coil,
           aspect=0,
           segment length=3,
           amplitude=4.0,
        }]  (3.1,2,1) -- (3.1,3,1);
        \draw [teal,thick,decorate, decoration={
            coil,
            aspect=0,
            segment length=6,
            amplitude=1.0,
        }]  (3.1,3,1) -- (3.1,4,1);
        \draw [teal,thick,decorate, decoration={
            coil,
            aspect=0,
            segment length=6,
            amplitude=2.0,
        }]  (3.1,4,1) -- (3.1,5,1);
        
        \draw[teal] (3.1,5,1) node [below] {{\small
            звук
        }};
        \end{tikzpicture}
     }
    \arrow[thick, rightarrow]{r}{}
    \&
    \raisebox{-.6\height}{
        \includegraphics[height=4cm]
            {img/video/video-signal.pdf}
    }
    \\
\end{tikzcd}
\end{frame}



\begin{frame}{Как превратить во временной ряд}



\begin{tikzpicture}[
    scale = 1.0,
    x={(-0.5cm,-0.04cm)},
    y={(0.3cm,-0.09cm)},
    z={(0cm,0.5cm)},
]
    \tikzset{normalframe/.style={
        fill=black!3,
        opacity=0.5,
        draw=black
    }}
    \begin{scope}[canvas is zx plane at y=0]
        \path[normalframe] (0,0) rectangle (2,3);
    \end{scope}
    \begin{scope}[canvas is zx plane at y=1]
        \path[normalframe] (0,0) rectangle (2,3);
    \end{scope}
    \begin{scope}[canvas is zx plane at y=2]
        \path[normalframe] (0,0) rectangle (2,3);
    \end{scope}
    \begin{scope}[canvas is zx plane at y=3]
        \path[normalframe] (0,0) rectangle (2,3);
    \end{scope}
    \begin{scope}[canvas is zx plane at y=4]
        \path[normalframe] (0,0) rectangle (2,3);
    \end{scope}
\end{tikzpicture}


    
\end{frame}