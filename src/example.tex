\section{Аномалия}

\subsection{Сигнал}

\begin{imageframe}{
    <<Дядя Стёпа~—~милиционер>>, внешний вид сигнала $F_{L_{1}}$
}
    \includegraphics[width=12cm]
        {img/video/example/original-signal.pdf}
\end{imageframe}

\begin{frame}{Что считаем аномалией}

    \graybox{Аномалия}{
        \begin{itemize}
            \item резкое изменение свойств наблюдаемого ряда
            \item в заранее неизвестный момент времени.
        \end{itemize}
        Иногда явление называют <<разладкой>>.
    }
    \vspace{2em}
    \orangebox{Как связано с событиями}{
        \begin{itemize}
            \item Моменты аномалий — моменты событий.
            \item В художественном видео:
            \begin{itemize}
                \item монтажные склейки или <<сцены>>;
                \item собственные события сюжета.
            \end{itemize}
        \end{itemize}
    }
\end{frame}

\section{Пороги}

\subsection{Статический}

\subsubsection{SAD}


\begin{imageframe}{
    <<Дядя Стёпа~—~милиционер>>, простой порог
 }
    \includegraphics[width=12cm]
    {img/video/example/static-treshold-sad.pdf}
\end{imageframe}


\subsubsection{FFMPEG}

\begin{imageframe}{
    <<Дядя Стёпа~—~милиционер>>, определение <<сцен>> в FFMPEG
}
    
    \includegraphics[width=12cm]
    {img/video/example/static-treshold-ffmpeg.pdf}
    
\end{imageframe}


\begin{imageframe}{
    <<Дядя Стёпа~—~милиционер>>, сравнение порогов
}
    \includegraphics[width=12cm]
        {img/video/example/static-treshold-both.pdf}
\end{imageframe}

\begin{imageframe}{Съёмка с БПЛА над побережьем Тулума (Мексика)}
    \includegraphics[width=12cm]%
        {img/video/example/static-treshold-sad-other.pdf}
\end{imageframe}


\begin{frame}{Пороговые методы cо статическим порогом}
    
\graybox{Как устроены}{
    \begin{itemize}
        \item разница соседних величин (по некоторой норме);
        \item превышения порога — <<аномалия>>.
    \end{itemize}%
}
\vspace{0.5em}
\orangebox{Плюсы}{
    \begin{itemize}
        \item просты в реализации, втч аппаратной;
        \item не требовательны к ресурсам.
    \end{itemize}%
}
\vspace{0.5em}
\bluebox{Минусы}{
    \begin{itemize}
        \item требуется заранее знать порог;
        \item не применимо для разных типов видео;
        \item чувствительны к случайным всплескам;
        \item ловят только краткосрочные события.
    \end{itemize}%
}
\end{frame}


\subsection{Адаптивный}

\begin{imageframe}{
        <<Дядя Стёпа~—~милиционер>>, \\
        адаптивный порог 
        $D_t > \hat{\mu}_{k} + A \cdot \hat{\sigma}_{k}$
    }
    \includegraphics[width=11cm]
    {img/video/example/sigma-treshold-ignore-last.pdf}
\end{imageframe}

\begin{imageframe}{
        Съёмка с БПЛА над побережьем Тулума, \\
        адаптивный порог 
        $D_t > \hat{\mu}_{k} + A \cdot \hat{\sigma}_{k}$
    }
    \includegraphics[width=11cm]
    {img/video/example/sigma-treshold-ignore-last-other.pdf}
\end{imageframe}

\begin{imageframe}{
        <<Дядя Стёпа~—~милиционер>>,
        адаптивный порог
        $D_t > \hat{\mu}_{k} + A \cdot \hat{\sigma}_{k}$ \\
        (не нашли плавный переход)
    }
    \includegraphics[width=11cm]
    {img/video/example/sigma-treshold-ignore-last-big.pdf}
\end{imageframe}


\begin{frame}{Пороговые методы адаптивным порогом}
    \graybox{Как устроены}{
        \begin{itemize}
            \item разница соседних величин (по некоторой норме);
            \item превышения порога — <<аномалия>>.
            \item порог вычисляется динамически (критерий Смирного-Граббса).
        \end{itemize}%
    }
    \vspace{0.5em}
    \orangebox{Плюсы}{
        \begin{itemize}
            \item не требовательны к ресурсам.
        \end{itemize}%
    }
    \vspace{0.5em}
    \bluebox{Минусы}{
        \begin{itemize}
            \item требуется заранее подобрать размер скользящего окна;
            \item не применимо для разных типов видео;
            \item чувствительны к случайным всплескам;
            \item ловят только краткосрочные события;
        \end{itemize}%
    }
\end{frame}


\section[Средние]{Сравнение средних}

\begin{imageframe}{
        <<Дядя Стёпа~—~милиционер>>, разницы средних\\
    }
    \includegraphics[width=11cm]
    {img/video/example/signal-mean-diff.pdf}
\end{imageframe}


\begin{imageframe}{
        <<Дядя Стёпа~—~милиционер>>, разницы средних \\
        (нашли центральную точку перехода)
    }
    \includegraphics[width=11cm]
    {img/video/example/signal-mean-diff-big.pdf}
\end{imageframe}


\begin{imageframe}{
        Съёмка с БПЛА над побережьем Тулума, разницы средних \\
    }
    \includegraphics[width=11cm]
    {img/video/example/signal-mean-diff-other.pdf}
\end{imageframe}


\begin{frame}{Сравнение средних}
    \graybox{Как устроены}{
        \begin{itemize}
            \item разница двух оценок сигнала;
            \item сигнал оцениваем через средние.
        \end{itemize}%
    }
    \vspace{0.5em}
    \orangebox{Плюсы}{
        \begin{itemize}
            \item не очень требовательны к ресурсам.
            \item могут ловить и резкие и плавные переходы
        \end{itemize}%
    }
    \vspace{0.5em}
    \bluebox{Минусы}{
        \begin{itemize}
            \item требуется заранее подобрать размеры скользящих окон;
            \begin{itemize}
                \item при малом размере ловят шум;
                \item при большом — могут пропустить событие;
            \end{itemize}%
            \item требуется хранить скользящее окно.
        \end{itemize}%
    }
\end{frame}

\section{Штрафы}

\begin{frame}{Что предлагаем}
    
    \graybox{Штрафы по производной средней оценки}{

        \begin{itemize}
            \item оценим сигнал кусочно-постоянной функцией $DTR_{w,d}(t)$:
            \begin{itemize}
                \item используем решающее дерево регрессии глубины $d$;
                \item по скользящему окну размера $w$;

            \end{itemize}
            \item размножение оценок:
            \begin{itemize}
                \item усредненная оценка по~$k$~регрессиям 
                $S_{DTR,d}(t) = \dfrac{1}{k}\sum\limits_{i=1}^{k} DTR_{i \cdot s,d}(t)$;
                \item окна разного размера $w = i \cdot s$, $s$ — шаг размера окна;
            \end{itemize}
            \item разница соседних точек:
            \begin{itemize}
                \item $S'_{DTR}(t) = \dfrac{d\,S_{DTR}(t)}{d\,t} = S_{DTR}(t) - S_{DTR}(t-1)$;
            \end{itemize}
            \item штраф за отклонение:
            \begin{itemize}
                 \item $B_{x}(t) = \left( |S'_{DTR}(t)| > |\hat{\mu}_{x}(S'_{DTR}(t)) + A \cdot \hat{\sigma}_{x}(S'_{DTR}(t))| \right) \in \left\lbrace 0,1 \right\rbrace$, 
                 \item $x$ — размер скользящего окна
            \end{itemize}
            \item еще раз размножение оценок (голосование):
            \begin{itemize}
                \item $V(t) = \dfrac{1}{n}\sum\limits_{j=1}^{n} B_{j \cdot z}(t)$;\quad $z$ — шаг размера окна;
            \end{itemize}
        \end{itemize}%

    }
\end{frame}


\begin{imageframe}{
        <<Дядя Стёпа~—~милиционер>>,
        штрафы по дискретной производной\\
        усредненной кусочно-постоянной аппроксимации
    }\\
    $V = \dfrac{1}{k}\sum\limits_{j=1}^{k} Bill\,S_{DTR} =  \dfrac{1}{k}\sum\limits_{j=1}^{k}(|S'_{DTR}| > |\hat{\mu}_{j}(S'_{DTR})+ A \cdot \hat{\sigma}_{j}(S'_{DTR})|)$
    \includegraphics[width=10cm]
    {img/video/example/dtr-treshold-bills.pdf}
\end{imageframe}


\begin{imageframe}{
        Съёмка с БПЛА над побережьем Тулума,
        штрафы по дискретной производной\\
        усредненной кусочно-постоянной аппроксимации
    }\\
    $V = \dfrac{1}{k}\sum\limits_{j=1}^{k} Bill\,S_{DTR} =  \dfrac{1}{k}\sum\limits_{j=1}^{k}(|S'_{DTR}| > |\hat{\mu}_{j}(S'_{DTR})+ A \cdot \hat{\sigma}_{j}(S'_{DTR})|)$
    \includegraphics[width=10cm]
    {img/video/example/dtr-treshold-bills-other.pdf}
\end{imageframe}



\begin{frame}{Что предлагаем}
    \orangebox{Плюсы}{
        \begin{itemize}
            \item интуитивный поиск аномалий;
            \item ловит и резкие и плавные переходы;
            \item относительный порог аномалии 
                можно выбрать после вычислений (в пределах).
        \end{itemize}%
    }
    \vspace{1em}
    \bluebox{Минусы}{
        \begin{itemize}
            \item требования вычислительным к ресурсам;
            \item пока плохо изучен, нужно исследовать свойства.
        \end{itemize}%
    }
\end{frame}


\section{Итог}

\begin{frame}{Итог}
    \graybox{Чего достигли}{
        \begin{itemize}
            \item предложен и реализован алгоритм поиска аномалий видео;
            \item по аномалия — определяем, 
            что в видео произошло событие
            \item проведены эксперименты;
            \item результаты совпадают с тем, как разметил человек.
        \end{itemize}%
    }
    \vspace{0.5em}    
    \graybox{Куда двигаться дальше}{
        \begin{itemize}
            \item получить размеченную выборку видео-событий;
            \item сравнить различные методы по этой выборке;
            \item получить коэффициенты доверия для конкретных методов;
            \item получить комбинированный метод.
        \end{itemize}%
    }
\end{frame}
