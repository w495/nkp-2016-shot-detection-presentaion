\section{Аномалия}

\subsection{Сигнал}

\begin{imageframe}
    \includegraphics[width=\paperwidth]
        {img/video/example/original-signal.pdf}
\end{imageframe}

\begin{frame}{Что считаем аномалией}

    \graybox{Аномалия}{
        \begin{itemize}
            \item резкое изменение свойств наблюдаемого ряда
            \item в заранее неизвестный момент времени.
        \end{itemize}
        Иногда явление называют <<разладкой>>.
    }
    \vspace{2em}
    \orangebox{Как связано с событиями}{
        \begin{itemize}
            \item Моменты аномалий — моменты событий.
            \item В художественном видео:
            \begin{itemize}
                \item монтажные склейки;
                \item собственные события сюжета.
            \end{itemize}
        \end{itemize}
    }
\end{frame}

\section{Пороги}

\subsection{SAD}

\begin{imageframe}
    \includegraphics[width=\paperwidth]
    {img/video/example/static-treshold-sad.pdf}
\end{imageframe}


\subsection{FFMPEG}

\begin{imageframe}

    \includegraphics[width=\paperwidth]
    {img/video/example/static-treshold-ffmpeg.pdf}
    
\end{imageframe}


\begin{imageframe}
    
\includegraphics[width=\paperwidth]
{img/video/example/static-treshold-both.pdf}

\end{imageframe}


\begin{frame}{Пороговые методы}
    
\graybox{Как устроены}{
    \begin{itemize}
        \item Вычисляется разница соседних величин по некоторой норме;
        \item Все значения, которые превышают порог — считаются <<аномальными>>.
    \end{itemize}%
}

\end{frame}

