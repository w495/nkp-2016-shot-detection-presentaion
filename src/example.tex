\section{Аномалия}

\subsection{Сигнал}

\begin{imageframe}{
    <<Дядя Стёпа~—~милиционер>>, внешний вид сигнала $F_{L_{1}}$
}
    \includegraphics[width=12cm]
        {img/video/example/original-signal.pdf}
\end{imageframe}

\begin{frame}{Что считаем аномалией}

    \graybox{Аномалия}{
        \begin{itemize}
            \item резкое изменение свойств наблюдаемого ряда
            \item в заранее неизвестный момент времени.
        \end{itemize}
        Иногда явление называют <<разладкой>>.
    }
    \vspace{2em}
    \orangebox{Как связано с событиями}{
        \begin{itemize}
            \item Моменты аномалий — моменты событий.
            \item В художественном видео:
            \begin{itemize}
                \item монтажные склейки или <<сцены>>;
                \item собственные события сюжета.
            \end{itemize}
        \end{itemize}
    }
\end{frame}

\section{Пороги}

\subsection{SAD}

\begin{imageframe}{
    <<Дядя Стёпа~—~милиционер>>, простой порог
 }
    \includegraphics[width=12cm]
    {img/video/example/static-treshold-sad.pdf}
\end{imageframe}


\subsection{FFMPEG}

\begin{imageframe}{
    <<Дядя Стёпа~—~милиционер>>, определение <<сцен>> в FFMPEG
}
    
    \includegraphics[width=12cm]
    {img/video/example/static-treshold-ffmpeg.pdf}
    
\end{imageframe}


\begin{imageframe}{
    <<Дядя Стёпа~—~милиционер>>, сравнение порогов
}
    \includegraphics[width=12cm]
        {img/video/example/static-treshold-both.pdf}
\end{imageframe}

\begin{imageframe}{Съёмка с БПЛА над побережьем Тулума (Мексика)}
    \includegraphics[width=12cm]%
        {img/video/example/static-treshold-sad-other.pdf}
\end{imageframe}


\begin{frame}{Пороговые методы}
    
\graybox{Как устроены}{
    \begin{itemize}
        \item Вычисляется разница соседних величин по некоторой норме;
        \item Все значения, которые превышают порог — считаются <<аномальными>>.
    \end{itemize}%
}
\end{frame}


\begin{imageframe}{
        <<Дядя Стёпа~—~милиционер>>, \\
        адаптивный порог 
        $D_t > \hat{\mu}_{k} + A \cdot \hat{\sigma}_{k}$
    }
    \includegraphics[width=11cm]
    {img/video/example/sigma-treshold-ignore-last.pdf}
\end{imageframe}

\begin{imageframe}{
        Съёмка с БПЛА над побережьем Тулума, \\
        адаптивный порог 
        $D_t > \hat{\mu}_{k} + A \cdot \hat{\sigma}_{k}$
    }
    \includegraphics[width=11cm]
    {img/video/example/sigma-treshold-ignore-last-other.pdf}
\end{imageframe}

\begin{imageframe}{
        <<Дядя Стёпа~—~милиционер>>, \\
        адаптивный порог
        $D_t > \hat{\mu}_{k} + A \cdot \hat{\sigma}_{k}$
    }
    \includegraphics[width=11cm]
    {img/video/example/sigma-treshold-ignore-last-big.pdf}
\end{imageframe}








