\begin{frame}{Сведение к одномерному случаю}

\graybox{Зачем сводить к одномерному случаю}{
    Сможем в явном виде применить:
    \begin{itemize}
        \item методы экономического анализа;
        \item методы обработки временных рядов из радиотехники;
        \item методы поиска разладок.
    \end{itemize}
}
\vspace{1em}
\greenbox{Почему сведение корректно}{
    \begin{itemize}
        \item можем свести из $n$-мерного:
        \begin{itemize}
            \item нужны только весовые коэффициенты;
            \item пример: $
                Y_{Y'P_rP_b} 
                    = 0.299 \cdot R + 0.587 \cdot G + 0.114  \cdot B;
            $
        \end{itemize}
        \item сможем обобщить до $n$-мерного:
        \begin{itemize}
            \item для визуальной информации $n$ может быть $> 3$;
            \item очень актуально для трехмерных камер.
        \end{itemize}

    \end{itemize}
}


\end{frame}