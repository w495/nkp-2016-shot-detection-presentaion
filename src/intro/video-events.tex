
\newcommand{\epath}{./img/video/events}

\newcommand{\eventexample}[1]{
    \includegraphics[width=3.0cm]{\epath/Djadja-Stepa-Milicioner-#1.png}
}


\begin{frame}{Что такое видео? — Последовательность ...}


\graybox{Видео — последовательность фактов или событий}{
    \begin{itemize}
        \item События развиваются во времени.
        \item Свойства событий
        — {\it пространственная} характеристика видео,
        \item продолжительность и порядок фактов — {\it временная}.
    \end{itemize}%
}
\vspace{1.5em}

Пример:
\begin{center}
    \begin{tikzcd}[column sep=0.02pc,row sep=0.3pc,ampersand replacement=\&]
        \raisebox{-.4\height}{\eventexample{1}}
             \raisebox{\height}{$\ ;$}
        \&
        \raisebox{-.4\height}{\eventexample{2}}
             \raisebox{\height}{$\ ;$}
        \&
        \raisebox{-.4\height}{\eventexample{3}}
              \raisebox{\height}{$\ .$}
        \\
    \end{tikzcd}
    \footnotesize
    \color{zdarkgreen} 
    (Кадры событий из начала мультфильма <<Дядя Стёпа~—~милиционер>>)
    
\end{center}%

\note{
    sdsd
    sdsd
}



%\graybox{Что есть факты?}{
%    \begin{itemize}
%        \item простейший вариант — сцены-съёмки:
%        \begin{itemize}
%            \item[$\triangleright$] ищем точки смены съёмок (монтажных склеек).
%        \end{itemize}
%        \item пространственная характеристика:
%        \begin{itemize}
%            \item[$\triangleright$] начальный и конечный кадр (<<мешок слов>>, GIST)
%        \end{itemize}
%        \item временная характеристика:
%        \begin{itemize}
%            \item[$\triangleright$] отношения длин съёмок к длинам соседних съёмок
%            \item[$\triangleright$] $\color{teal}+$ алгоритмы выравнивания последовательностей
%        \end{itemize}
%    \end{itemize}
%}


\end{frame}
