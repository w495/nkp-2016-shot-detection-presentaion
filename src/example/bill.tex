
\begin{frame}{Что предлагаем}
    
    \graybox{Штрафы по производной средней оценки}{
        
        \begin{itemize}
            \item оценим сигнал кусочно-постоянной функцией $DTR_{w,d}(t)$:
            \begin{itemize}
                \item используем решающее дерево регрессии глубины $d$;
                \item по скользящему окну размера $w$;
                
            \end{itemize}
            \item размножение оценок:
            \begin{itemize}
                \item усредненная оценка по~$k$~регрессиям 
                $S_{d}(t) = \dfrac{1}{k}\sum\limits_{i=1}^{k+1} DTR_{i \cdot y,d}(t)$;
                \item окна разного размера $w = i \cdot g$, $g$ — шаг размера окна;
            \end{itemize}
            \item разница соседних точек:
            \begin{itemize}
                \item $S'(t) = \dfrac{d\,S(t)}{d\,t} = S(t) - S(t-1)$;
            \end{itemize}
            \item штраф за отклонение:
            \begin{itemize}
                \item $B_{w}(t) = \left( |S'(t)| > |\hat{\mu}_{w}(S'(t)) + A \cdot \hat{\sigma}_{w}(S'(t))| \right) \in \left\lbrace 0,1 \right\rbrace$;
            \end{itemize}
            \item еще раз размножение оценок (голосование):
            \begin{itemize}
                \item $V(t) = \dfrac{1}{n}\sum\limits_{j=1}^{n+1} B_{j \cdot z}(h)$;\quad $h$ — шаг размера окна;
            \end{itemize}
        \end{itemize}%
        
    }
\end{frame}

\begin{frame}[fragile]{DSL}

Для решения задачи мы разработали мета-язык на базе языка Python.
\begin{lstlisting}[language=FilterPython]
delay = DelayFilter()      # %{\cmnt Фильтр линейной задержки.}%
original = delay(0)        # %{\cmnt Входящий сигнал без изменения.}%
shift = ShiftSWFilter()    # %{\cmnt Сдвиг сигнала на один кадр.}%
diff = original - shift    # %{\cmnt Разница соседних кадров.}%
norm = NormFilter()        # %{\cmnt Норма сигнала.}%
modulus = ModulusFilter()  # %{\cmnt Модуль сигнала.}%
mean = MeanSWFilter()      # %{\cmnt Среднее по скользящему окну.}%
std = StdSWFilter()        # %{\cmnt Стандартное отклонение по окну.}%
dtr = DecisionTreeRegressorSWFilter()

n=8, k=8, h=25, g=25, d=2, a=3.0 # %{\cmnt параметры алгоритма}%

S = sum(dtr(w=s*i+1,d=d) for i in range(1,k + 1)) / k
B = lambda w: delay(0) > (mean(w=w) + a * std(w=w))
V = sum(B(w=z*j) for j in range(1,n + 1)) / n

# %{\cmnt Фильтр: собирается отложено до непосредсвенного применения.}%
# %{\cmnt Оператор <<|>> означает <<конвейер>>.}%
vfilter = norm(l=1) | S | diff | modulus | V
\end{lstlisting}
\end{frame}

\begin{imageframe}{
        <<Дядя Стёпа~—~милиционер>>,
        штрафы по дискретной производной\\
        усредненной кусочно-постоянной аппроксимации
    }
    \includegraphics[width=11cm]
    {img/video/example/bill/dtr-stepa-from-00.pdf}
\end{imageframe}


\begin{imageframe}{
        Съёмка с БПЛА над побережьем Тулума,
        штрафы по дискретной производной\\
        усредненной кусочно-постоянной аппроксимации
    }\\
    \includegraphics[width=11cm]
    {img/video/example/bill/dtr-tulum.pdf}
\end{imageframe}



\begin{frame}{Что предлагаем}
    \orangebox{Плюсы}{
        \begin{itemize}
            \item интуитивный поиск аномалий;
            \item ловит и резкие и плавные переходы;
            \item относительный порог аномалии 
            можно выбрать после вычислений (в пределах).
        \end{itemize}%
    }
    \vspace{1em}
    \bluebox{Минусы}{
        \begin{itemize}
            \item требования вычислительным к ресурсам;
            \item пока плохо изучен, нужно исследовать свойства.
        \end{itemize}%
    }
\end{frame}
