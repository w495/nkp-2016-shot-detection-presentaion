
\begin{frame}{Что предлагаем}
    
    \graybox{Штрафы по производной средней оценки}{
        
        \begin{itemize}
            \item оценим сигнал кусочно-постоянной функцией $DTR_{w,d}(t)$:
            \begin{itemize}
                \item используем решающее дерево регрессии глубины $d$;
                \item по скользящему окну размера $w$;
                
            \end{itemize}
            \item размножение оценок:
            \begin{itemize}
                \item усредненная оценка по~$k$~регрессиям 
                $S_{d}(t) = \dfrac{1}{k}\sum\limits_{i=1}^{k} DTR_{i \cdot s,d}(t)$;
                \item окна разного размера $w = i \cdot s$, $s$ — шаг размера окна;
            \end{itemize}
            \item разница соседних точек:
            \begin{itemize}
                \item $S'(t) = \dfrac{d\,S(t)}{d\,t} = S(t) - S(t-1)$;
            \end{itemize}
            \item штраф за отклонение:
            \begin{itemize}
                \item $B_{x}(t) = \left( |S'(t)| > |\hat{\mu}_{x}(S'(t)) + A \cdot \hat{\sigma}_{x}(S'(t))| \right) \in \left\lbrace 0,1 \right\rbrace$, 
                \item $x$ — размер скользящего окна
            \end{itemize}
            \item еще раз размножение оценок (голосование):
            \begin{itemize}
                \item $V(t) = \dfrac{1}{n}\sum\limits_{j=1}^{n} B_{j \cdot z}(t)$;\quad $z$ — шаг размера окна;
            \end{itemize}
        \end{itemize}%
        
    }
\end{frame}


\begin{imageframe}{
        <<Дядя Стёпа~—~милиционер>>,
        штрафы по дискретной производной\\
        усредненной кусочно-постоянной аппроксимации
    }
    \includegraphics[width=11cm]
    {img/video/example/bill/dtr-stepa-from-00.pdf}
\end{imageframe}


\begin{imageframe}{
        Съёмка с БПЛА над побережьем Тулума,
        штрафы по дискретной производной\\
        усредненной кусочно-постоянной аппроксимации
    }\\
    \includegraphics[width=11cm]
    {img/video/example/bill/dtr-tulum.pdf}
\end{imageframe}



\begin{frame}{Что предлагаем}
    \orangebox{Плюсы}{
        \begin{itemize}
            \item интуитивный поиск аномалий;
            \item ловит и резкие и плавные переходы;
            \item относительный порог аномалии 
            можно выбрать после вычислений (в пределах).
        \end{itemize}%
    }
    \vspace{1em}
    \bluebox{Минусы}{
        \begin{itemize}
            \item требования вычислительным к ресурсам;
            \item пока плохо изучен, нужно исследовать свойства.
        \end{itemize}%
    }
\end{frame}
