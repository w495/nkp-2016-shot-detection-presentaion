
\begin{frame}{Пороговые методы cо статическим порогом}
    
    \graybox{Как устроены}{
        \begin{itemize}
            \item разница соседних величин (по некоторой норме);
            \item превышения порога — <<аномалия>>.
        \end{itemize}%
    }
    \vspace{0.5em}
    \orangebox{Плюсы}{
        \begin{itemize}
            \item просты в реализации, втч аппаратной;
            \item не требовательны к ресурсам.
        \end{itemize}%
    }
    \vspace{0.5em}
    \bluebox{Минусы}{
        \begin{itemize}
            \item требуется заранее знать порог;
            \item не применимо для разных типов видео;
            \item чувствительны к случайным всплескам;
            \item ловят только краткосрочные события.
        \end{itemize}%
    }
\end{frame}

\subsubsection{SAD}

\begin{imageframe}{
    <<Дядя Стёпа~—~милиционер>>, простой порог
 }
    \includegraphics[width=12cm]
    {img/video/example/static-treshold-sad.pdf}
\end{imageframe}


\subsubsection{FFMPEG}

\begin{imageframe}{
    <<Дядя Стёпа~—~милиционер>>, определение <<сцен>> в FFMPEG
}
    
    \includegraphics[width=12cm]
    {img/video/example/static-treshold-ffmpeg.pdf}
    
\end{imageframe}


\subsubsection{Сравнение порогов}

\begin{imageframe}{
    <<Дядя Стёпа~—~милиционер>>, сравнение порогов
}
    \includegraphics[width=12cm]
        {img/video/example/static-treshold-both.pdf}
\end{imageframe}


\subsubsection{Чем плох статический порог}

\begin{imageframe}{Съёмка с БПЛА над побережьем Тулума (Мексика)}
    \includegraphics[width=12cm]%
        {img/video/example/static-treshold-sad-other.pdf}
\end{imageframe}

