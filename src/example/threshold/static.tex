
\begin{frame}{Пороговые методы cо статическим порогом}
    \graybox{Как устроены}{
        \begin{itemize}
            \item разница соседних величин (по некоторой норме);
            \item превышения порога — <<аномалия>>.
        \end{itemize}%
    }
    \vspace{0.5em}
    \orangebox{Плюсы}{
        \begin{itemize}
            \item просты в реализации, втч аппаратной;
            \item не требовательны к ресурсам.
        \end{itemize}%
    }
    \vspace{0.5em}
    \bluebox{Минусы}{
        \begin{itemize}
            \item требуется заранее знать порог;
            \item не применимо для разных типов видео;
            \item чувствительны к случайным всплескам;
            \item ловят только краткосрочные события.
        \end{itemize}%
    }
\end{frame}

\subsubsection{Разница по модулю}

\begin{imageframe}{
    <<Дядя Стёпа~—~милиционер>>, простой порог
 }
    \includegraphics[width=12cm]
    {img/video/example/threshold/static/sad-stepa.pdf}
\end{imageframe}

\note{
    \begin{normalsize}
        В качестве исследуемой величины тут используется 
        значение нормированной попиксельной 
        абсолютной разницы яркостей двух соседних кадров.
        В качестве нормы используется векторная норма $L_1$.
        \[
            D_t            
                = \left\| F_t - F_{t-1} \right\|_{L_1} 
                    = \dfrac{1}{n \cdot m} 
                        \sum\limits_{i=0}^{n}
                        \sum\limits_{j=0}^{m}
                            \left| x_{t,i,j} - x_{t-1,i,j} \right| 
        \]
        \begin{itemize}
            \item $x_{t,i,j}$ — пиксель кадра $F_t$;
            \item $x_{t-1,i,j}$ — пиксель кадра $F_{t-1}$;
        \end{itemize}
        Физический смысл этой величины 
        $\left| x_{t,i,j} - x_{t-1,i,j} \right|$
        заключается в том, насколько один пиксель отличается от другого.
    \end{normalsize}
}

\begin{frame}[fragile]{Простой порог: Мета-язык потоковой обработки}
    \begin{lstlisting}[language=FilterPython]
    delay = DelayFilter()      # %{\cmnt Фильтр линейной задержки.}%
    orig  = delay(0)           # %{\cmnt Входящий сигнал без изменения.}%
    shift = ShiftSWFilter()    # %{\cmnt Сдвиг сигнала на один кадр.}%
    diff = orig - shift        # %{\cmnt Разница соседних кадров.}%
    norm = NormFilter()        # %{\cmnt Норма сигнала.}%
    modulus = ModulusFilter()  # %{\cmnt Модуль сигнала.}%
    T_CONST = 0.08             # %{\cmnt Значение порога. Константа.}%
    threshold = orig > T_CONST # %{\cmnt Пороговый фильтр.}%
    
    # %{\cmnt Фильтр нормированной попиксельной абсолютной разницы.}%
    d_filter = diff | modulus | norm(l=1)
    
    # %{\cmnt Результирующий фильтр: точки разладок.}%
    result_filter = d_filter | threshold
    \end{lstlisting}
    \begin{itemize}
        \item Оператор <<|>> означает <<конвейер>>.
        \item Фильтры собирается отложено 
                до непосредственного применения.
    \end{itemize}

\end{frame}

\note{
    \begin{normalsize}
        Для решения задачи мы разработали мета-язык 
        на базе языка Python.\\
        Свойства мета-языка:
        \begin{enumerate}
            \item Основной сущностью является фильтр:
            \begin{itemize}
                \item по-факту это просто абстрактный тип 
                на языке Python с перегруженными операторвами,
                \item фильтры описывают набор действий 
                над последовательностью кадров видео;
            \end{itemize}
            \item Любые прочие объекты приводятся к фильтру и действия с ними осуществляются как с фильтром;
            \item Жизненный цикл фильтра состоит из двух этапов:
            \begin{itemize}
                \item объявление фильтра — при создании фильтра никаких вычислений над последовательностью кадров не происходит,
                \item выполнении фильтра —
                (вызов метода класса $filter\_objects$) 
                внутри фильтра происходит обработка 
                последовательности объектов: 
                кадров или результатов работы других фильтров;
            \end{itemize}
        \end{enumerate}
    \end{normalsize}
}

\note{
    \begin{normalsize}
        Операции над фильтрами:
        \begin{enumerate}
            \item Любая операция над фильтрами кроме выполнения,
            приводит к созданию нового фильтра:
            \begin{itemize}
                \item на этапе выполнения фильтра будут вычислены 
                аргументы этой операции и выполнена сама операция;
                \item порядок вычисления аргументов операций не определен, и на многоядерных архитектурах аргументы операций могут быть вычисленны параллельно;
            \end{itemize}
            \item Основная операция над фильтрами — последовательное их применение — конвейер. Она тоже приводит к созданию нового фильтра.
            \item Прочие операции над фильтрами приводят 
            к изменению последовательности объектов или их свойств
            в процессе выполнения итогового фильтра.
        \end{enumerate}
    \end{normalsize}
}



\subsubsection{FFMPEG}

\begin{imageframe}{
    <<Дядя Стёпа~—~милиционер>>, определение <<сцен>> в FFMPEG
}
    \includegraphics[width=12cm]
    {img/video/example/threshold/static/ffmpeg-stepa.pdf}
    
\end{imageframe}

\note{
Обратите внимание, что в случае FFMPEG, 
в качетсве велич
}


\subsubsection{Сравнение порогов}

\begin{imageframe}{
    <<Дядя Стёпа~—~милиционер>>, сравнение порогов
}
    \includegraphics[width=12cm]
        {img/video/example/threshold/static/both-stepa.pdf}
\end{imageframe}


\subsubsection{Чем плох статический порог}

\begin{imageframe}{Съёмка с БПЛА над побережьем Тулума (Мексика)}
    \includegraphics[width=12cm]%
        {img/video/example/threshold/static/sad-tulum.pdf}
\end{imageframe}

\begin{frame}{Чем плох статический порог}
    
    \bluebox{Что получилось:}{ 
        Для определения событий в~Съёмке с~БПЛА над~побережьем Тулума,
        порога $0.08$ не хватило.
        \begin{itemize}
            \item Все разницы кадров оказались меньше порога.
            \item События найдены не были.
        \end{itemize}
    }
    \vspace{1.5em}
    \orangebox{В чём проблема:}{
        \begin{itemize}
            \item величину порога надо знать заранее;
            \item нельзя использовать везде одну и ту же величину:
            \begin{itemize}
                \item в плавном видео — малые разницы кадров;
                \item в динамичном видео — большие разницы кадров.
            \end{itemize}
        \end{itemize}%
    }
\end{frame}


\begin{frame}{Улучшения статического порога}
    \orangebox{Проблема:}{
        \begin{itemize}
        \item большой порог для плавного видео:
            \begin{itemize}
                \item события найдены не будут;
            \end{itemize}
        \item маленький порог для динамичного видео:
            \begin{itemize}
                \item много ложных срабатываний;
            \end{itemize}
        \item если видео имеет плавные и динамичные участки:
            \begin{itemize}
                \item порог подобрать невозможно.
            \end{itemize}
        \end{itemize}
    }
    \vspace{1.5em}
    \greenbox{Решение:}{
        \begin{itemize}
            \item Учитывать «динамичность» видео:
            \begin{enumerate}
                \item Масштабировать разницы кадров:
                \begin{itemize}
                    \item нормировать по окрестности.
                \end{itemize}
                \item Учитывать среднюю величину и дисперсию.
            \end{enumerate}
        \end{itemize}
    }
\end{frame}



\begin{frame}{Масштабировать разницы кадров}


\[
    Norm_{n} = \dfrac
    {F_{t} - \min \left(  F_{t-n} \dots  F_{t} \right)}
    {\max \left( F_{t-n} \dots  F_{t} \right)
        - \min \left( F_{t-n} \dots  F_{t} \right)}
\]    
\end{frame}



\begin{frame}[fragile]{Масштабировать разницы кадров: Мета-язык}
\begin{footnotesize} 
Мета-язык на базе языка Python.
\end{footnotesize}
\begin{lstlisting}[language=FilterPython]
delay = DelayFilter()      # %{\cmnt Фильтр линейной задержки.}%
orig  = delay(0)           # %{\cmnt Входящий сигнал без изменения.}%
shift = ShiftSWFilter()    # %{\cmnt Сдвиг сигнала на один кадр.}%
diff = orig - shift        # %{\cmnt Разница соседних кадров.}%
norm = NormFilter()        # %{\cmnt Норма сигнала.}%
modulus = ModulusFilter()  # %{\cmnt Модуль сигнала.}%

# %{\cmnt Возвращает скользящее окно размер 200 для каждого кадра}%
sw = BaseSWFilter(s=200, min_size=2)

swmax = sw | max  # %{\cmnt Максимум по скользящему окну.}%
swmin = sw | min  # %{\cmnt Минимум по скользящему окну.}%

swnorm = (orig - swmin) / (swmax - swmin) # %{\cmnt Масштабирование.}%

# %{\cmnt Результирующий фильтр}%
result_filter = norm(l=1) | diff | modulus | swnorm
\end{lstlisting}
\end{frame}

\note{
Для решения задачи мы разработали мета-язык на базе языка Python.
Фильтры собирается отложено до непосредственного применения.
Оператор <<|>> означает <<конвейер>>.
}

\begin{imageframe}{
Съёмка с БПЛА над побережьем Тулума,
масштабированный простой порог
}
    \includegraphics[width=12cm]%
    {img/video/example/threshold/static/sad-swnorm-200-tulum.pdf}
\end{imageframe}

